% Chapter Template

\chapter{Conclusions} % Main chapter title

\label{Chapter:Conclusions} % Change X to a consecutive number; for referencing this chapter elsewhere, use \ref{ChapterX}

%----------------------------------------------------------------------------------------
%	SECTION 1
%----------------------------------------------------------------------------------------

\section{Summary of contributions}

Medical diagnosis is heavily supported by Medical Imaging. The analysis of medical images requires highly specialized expertise that is provided by specialized doctors. The advent of advanced machine learning techniques, like deep learning, is facilitating the design of high performance automatic classifiers in a broad range of applications, also for medical diagnosis. The purpose of this thesis is the exploration of new automatic diagnostic methods for medical diagnosis, concretely for diabetic retinopathy disease grading. As stated before in this work, DR is one of the main causes of blindness in the world. Early detection can reduce disease progression and consequently also the incidence of blindness. The diagnostic of DR is done primarily by retina fundus image analysis, that is done by ophthalmologists specifically trained for this purpose. Automatic diagnostic systems for DR can reduce dramatically not only the costs associated to diagnostic, but also the probability of developing blindness in the general population.

For this purpose we use supervised deep learning techniques. Given a class differentiation based on objective properties present in data, these parameterized models are able to learn the statistical regularities that have to be taken into account to separate the images. In this thesis, convolutional neural networks are used, which are efficient neural networks designed for exploiting local high correlations present in images. 

Machine learning methods in general and deep learning in particular are models that learn from data. Thus, a key factor for the success is having a statistically representative dataset of the population from which we want to predict a concrete property, in this case, a disease class. For this purpose it is required to have a labeled dataset with enough samples (that are of the order of magnitude of thousands elements per class) in order to create models with good generalization. For this purpose, we use a public dataset of EyePACS. This dataset is described in chapter \ref{Chapter:Background}.

In chapter \ref{Chapter:Classification}, we explore the first classification methods. In this case the original images have different high resolution values. Ideally, it is interesting to use images with the highest resolution available, but memory and computation time required make infeasible to use neural network models in such conditions. Furthermore, neural networks architectures are parameterized models that require the usage of constant input size. Additionally, lesions present in images, from which classification is inferred, can be detected with lower resolutions. In this chapter, we explore the usage of different input sizes, data augmentation techniques and optimization strategies, that serve as a basis for selecting the best hyper-parameters. Ensembling techniques, averaging a set of predictions coming from the evaluation of different rotated versions of the same input image, and bayesian analysis, combining predictions of one patient eye with conditional probabilities of having the disease in the other, are also used for improving results. From this preliminary work, a near human-level performance model is obtained. Loss function used in this chapter for neural network parameter optimization is the established standard for multi-class classification, ie. logarithmic loss.

As the evaluation function used for performance measure is quadratic weighted kappa, it is hypothesized that directly optimizing such function could generalize better. In chapter \ref{Chapter:QWK_loss} it is presented a way to use quadratic weighted kappa as a loss function for optimization of neural networks. First order derivatives are developed, in order to facilitate its usage with gradient descent derived optimization methods. Different ordinal regression use cases using quadratic weighted kappa as a main evaluation function are studied, proving that in all of them, more than 5\% increase in performance can be achieved optimizing directly QWK instead of logarithmic loss. 
%Afterwards, a new general purpose loss function was proposed for solving ordinal regression problems in deep learning.

In chapter \ref{Chapter:Ordinal_Regression} the new hardware available enables the usage of higher input resolutions and a faster computation time. This fact combined with the usage of hyper-parameters selected in chapter \ref{Chapter:Classification} and with the design of the new loss function done in chapter \ref{Chapter:QWK_loss} permits the design of models with ophthalmologist-level performance. In section \ref{class_guidelines} of this chapter, the most important hyper-parameters that have to be taken into account to achieve such results are summarized. This guidelines include parameters like the input size, convolution operators recommended sizes, number of filters per layer, data augmentation strategies, etc.

In chapter \ref{Chapter:Stability}, model robustness against changes in input images is also studied. Deep learning models are known to be very sensible to changes in input conditions. The objective of this chapter is the study of model robustness to typical changes in input conditions. The studied variables are rotation, hue, saturation and lightness, experimentally proving that the designed model is robust to changes in lightness, saturation and rotation but very sensible, as expected, to hue changes in input images. The hue dependency seems to be due to the fact that network use color information for differentiating lesions from healthy tissue.

At this point, performance standards are achieved, but having the model high statistical confidence, it shares with other deep learning models a major drawback, its lack of interpretability made it behave as a black box intuition machine. For improving medical applicability of such classifiers it is very important not only to have a high confidence statistical classifier but also to have a model that is able to provide an interpretation of the results to physicians.

With the idea of providing interpretation capabilities to our models, in chapter \ref{Chapter:Interpretation}, a general purpose pixel-wise explanation model is designed. Such interpretation model backpropagates last layer classification scores until reaching input space generating the so called pixel score maps, that are maps with the distribution of the final score for each input pixel. The pixels having higher scores are identified as the most important for a particular classification. Such model propagates scores through the network in two different ways: scores that depend on layer input are directly propagated to previous layer and constant scores of the layer, ie. related to biases, are transferred directly to the corresponding receptive field using a gaussian prior. Such interpretation model is designed in a modular manner, making it usable for any model architecture and in any application. In this thesis, its validity for predicting lesions (that explain classifications for the diabetic retinopathy disease grading) is tested, achieving high correlation between predicted and real lesions. 

In chapter \ref{Chapter:ICA}, another way of improving interpretabilibity is explored, that is feature space data compression. Feature space deep learning models tend to be high correlated, high dimensional spaces. In this chapter, a method for compression of such space is proposed. In cases where classification layer is designed as a linear feature combinations we assume that the model is able to disentangle the important features required for the classification, making its dependency linear. For such models, a method is proposed for compressing the feature space based on Independent Component Analysis. A method for the selection of optimal number of components is proposed. The proposed method is of general applicability and its validity is tested using our diabetic retinopathy disease grading model. The original vector space of 64 dimensions is compressed to only three features, achieving 99\% of performance of the original model without compression. Furthermore, such independent components can independently visualized in the input space using the method designed in chapter \ref{Chapter:Interpretation}.

Finally, in chapter \ref{Chapter:Inference} designed models are adapted to be used in a potentially different population coming from Hospital de Reus patients. Feature visualization techniques were used in order to check that the class separation capabilities showed in the original model are compatible with the new population, arriving to a positive conclusion. Compatibility of class definitions is also tested, detecting some discrepancies between the standards defined in EyePACS and the HUSJR requirements. A retraining of the classification layer using Messidor-2 dataset was enough to achieve ophtalmologist level accuracy in the new dataset. 

After the work done in this thesis, we can conclude that designed models can be used successfully as a high confidence diagnostic tool that can help to reduce costs in diagnostic and also to reduce the incidence of the DR disease in the general population.

\vspace{1cm} 

%-----------------------------------
%	SUBSECTION 1
%-----------------------------------
\section{Future research lines}

After the work done in this PhD thesis and the knowledge we have got from deep learning models as well as from diabetic retinopathy disease, some other study lines can be derived. 

Future research lines can be focused on the next different directions, that we explain below:

\begin{enumerate}

\item \emph{Increase the number of classes to predict}: One of the lines to explore is increasing the number of properties to predict of the original model. Instead of only predicting diabetic retinopathy class, other classes can also be included. Gradability class for example, can add information about image quality, ie. if its enough to be used as a proof of disease. Macular Edema prediction can also be added, incrementing the disease diagnostic information given by the automatic classifier. Furthermore, other diseases not directly related with diabetic retinopathy can also be detected, broadening the spectra of diagnostic. We hypothesize that using the same network for diagnostic of different diseases can help to increase the network performance, due to the increase the number of images available and to the added diversity for the detection of different diseases.

\item \emph{Transfer learning}: Successful networks in challenging tasks like ImageNet, can be good candidates to perform well in specialized medical imaging tasks like ours. Transferring knowledge from trained networks to other completely different fields have been proven to be a good way to initialize networks and improving performance. 

\item \emph{Unsupervised learning}: Generative Adversarial Networks can also be explored for generating new high quality samples from the original dataset. Such sample generation models are able to generate potentially unlimited number of artificial training samples that could allow, not only the training of more powerful models, but also the usage of artificially generated images that do not belong to any particular patient, ie. removing completely potential privacy legal concerns. 

\item \emph{Reinforcement Learning:} Adding to the models the possibility of enhancing its performance, designing online learning methods that allow continuous learning of networks from the corrections done by ophthalmologists on inference time. Hospital Universitari Sant Joan de Reus wants to use the classification model developed in this thesis in its daily work. This needs a previous process of integration of computer systems as well as the permission of Catalan Health Care authorities. If it is finally done, learning from online work could be very interesting.


%\item \emph{Adversarial examples}: The study of how minor modifications in input images alter model classifications and exploring new designs for increasing model robustness.

\item \emph{Use of Interpretation Model in other applications}: The designed interpretation model is domain independent, therefore, it would be interesting the study of its application in other domains, ie. for interpretation of other medical imaging classification tasks like for example, radiology, ultrasound, or other related imaging applications. 

\item \emph{Interpretation model for unsupervised image segmentation}: Other interesting aspect to explore is its usage in the segmentation domain. For this particular area of expertise U-Net derived networks are working very well but it could be interesting to explore the usage of classification networks amplified with its interpretation model as a alternative way of segmenting images without specifically annotating segmentation masks.

\end{enumerate}
