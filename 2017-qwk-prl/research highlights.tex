%% This is file `prletters-template.tex',
%% 
%% Copyright 2013 Elsevier Ltd
%% 
%% This file is part of the 'Elsarticle Bundle'.
%% ---------------------------------------------
%% 
%% It may be distributed under the conditions of the LaTeX Project Public
%% License, either version 1.2 of this license or (at your option) any
%% later version.  The latest version of this license is in
%%    http://www.latex-project.org/lppl.txt
%% and version 1.2 or later is part of all distributions of LaTeX
%% version 1999/12/01 or later.
%% 
%% The list of all files belonging to the 'Elsarticle Bundle' is
%% given in the file `manifest.txt'.
%% 
%% Template article for Elsevier's document class `elsarticle'
%% with harvard style bibliographic references
%%
%% $Id: prletters-template-with-authorship.tex 69 2013-07-15 10:15:25Z rishi $
%%
%% This template has no review option
%% 
%% Use the options `twocolumn,final' to obtain the final layout
\documentclass[times,twocolumn,final,authoryear]{elsarticle}

%% Stylefile to load PR Letters template
\usepackage{prletters}
\usepackage{framed,multirow}

%% The amssymb package provides various useful mathematical symbols
\usepackage{amssymb}
\usepackage{latexsym}

% Following three lines are needed for this document.
% If you are not loading colors or url, then these are
% not required.
\usepackage{url}
\usepackage{xcolor}
\definecolor{newcolor}{rgb}{.8,.349,.1}

%%%%%%%%%%%%%%%
% jdelatorre added header
\usepackage{tikz}
\usepackage{blindtext, graphicx}
\usepackage{smartdiagram}
\newcommand{\degree}{\ensuremath{^{\circ}}\xspace} % adds degree sign 
\usepackage[utf8]{inputenc}
\usepackage[T1]{fontenc} % adds capability to < sign
\usepackage{mathptmx}
\usepackage{amsmath}

% forced line break inside a tabular cell
\newcommand{\specialcell}[2][c]{%
	\begin{tabular}[#1]{@{}c@{}}#2\end{tabular}}

% use it for example as \specialcell[t]{Foo\\bar} or \specialcell{Foo\\bar}
%%%%%%%%%%%%%%%

\journal{Pattern Recognition Letters}

\begin{document}

\clearpage
\thispagestyle{empty}

\ifpreprint
  \vspace*{-1pc}
\else
%  \vspace*{-6pc}
\fi

\begin{table*}[!t]
\ifpreprint\else\vspace*{-15pc}\fi

\section*{Research Highlights (Required)}

To create your highlights, please type the highlights against each
\verb+\item+ command. 

\vskip1pc

\fboxsep=6pt
\fbox{
\begin{minipage}{.95\textwidth}
It should be short collection of bullet points that convey the core
findings of the article. It should  include 3 to 5 bullet points
(maximum 85 characters, including spaces, per bullet point.)  
\vskip1pc
\begin{itemize}

 \item Proposal of weighted kappa as loss function for ordinal regression in deep learning

 \item Derivation of equations required for applying first order optimization algorithms

 \item Solve a three use cases of ordinal regression using log and kappa losses

 \item Performance comparison of the three cases trained with both losses

 \item Check stability of the kappa loss function using different input and batch sizes

\end{itemize}
\vskip1pc
\end{minipage}
}

\end{table*}

\end{document}