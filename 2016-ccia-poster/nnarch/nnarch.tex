
\documentclass[crop,tikz]{standalone}
\usepackage{tikz} 
%%%<
\usepackage{verbatim}
\usepackage[active,tightpage]{preview}
\PreviewEnvironment{tikzpicture}
\setlength\PreviewBorder{5pt}%
%%%>

\usetikzlibrary{shadows,arrows}
% Define the layers to draw the diagram
\pgfdeclarelayer{background}
\pgfdeclarelayer{foreground}
\pgfsetlayers{background,main,foreground}
 
% Define block styles  
\tikzstyle{materia}=[draw, fill=blue!20, text width=6.0em, text centered,
  minimum height=1.5em,drop shadow]
\tikzstyle{practica} = [materia, text width=8em, minimum width=10em,
  minimum height=3em, rounded corners, drop shadow]
\tikzstyle{texto} = [above, text width=6em, text centered]
\tikzstyle{linepart} = [draw, thick, color=black!50, -latex', dashed]
\tikzstyle{line} = [draw, thick, color=black!50, -latex']
\tikzstyle{ur}=[draw, text centered, minimum height=0.01em]

\tikzstyle{convolution} = [materia, text width=8em, minimum width=10em,
minimum height=3em, rounded corners, drop shadow]
 
% Define distances for bordering
\newcommand{\blockdist}{1.3}
\newcommand{\edgedist}{1.5}

\newcommand{\practica}[2]{node (p#1) [practica]
  {Pr\'actica #1\\{\scriptsize\textit{#2}}}}

\newcommand{\convolution}[2]{node (p#1) [convolution]
	{Conv/BN/ReLU\\{\scriptsize\textit{#2}}}}
\newcommand{\full}[2]{node (p#1) [convolution]
	{FC BN ReLU Dropout\\{\scriptsize\textit{#2}}}}
\newcommand{\out}[2]{node (p#1) [convolution]
	{Output\\{\scriptsize\textit{#2}}}}

% Draw background
\newcommand{\background}[5]{%
  \begin{pgfonlayer}{background}
    % Left-top corner of the background rectangle
    \path (#1.west |- #2.north)+(-0.5,0.5) node (a1) {};
    % Right-bottom corner of the background rectanle
    \path (#3.east |- #4.south)+(+0.5,-0.25) node (a2) {};
    % Draw the background
    \path[fill=yellow!20,rounded corners, draw=black!50, dashed]
      (a1) rectangle (a2);
    \path (a1.east |- a1.south)+(0.8,-0.3) node (u1)[texto]
      {\scriptsize\textit{ #5}};
  \end{pgfonlayer}}

\newcommand{\transreceptor}[3]{%
  \path [linepart] (#1.east) -- node [above]
    {\scriptsize Transreceptor #2} (#3);}


\usepackage{helvet}
\renewcommand{\familydefault}{\sfdefault}

\begin{document}
\begin{tikzpicture}[scale=0.7,transform shape]
 
  % Draw diagram elements
  \path \convolution {1}{S:3x3, F:8};
  \path (p1.south)+(0.0,-1.0) \convolution{2}{S:3x3, F:16};
  \path (p2.south)+(0.0,-1.0) \convolution{3}{S:3x3 F:16};
  \path (p3.south)+(0.0,-1.5) \convolution{4}{S:3x3, F:32};
  \path (p4.south)+(0.0,-1.0) \convolution{5}{S:3x3 F:32};
  \path (p5.south)+(0.0,-1.5) \convolution{6}{S:3x3, F:64};
  \path (p6.south)+(0.0,-1.0) \convolution{7}{S:3x3 F:64};
  \path (p7.south)+(0.0,-1.5) \convolution{8}{S:3x3, F:128};
  \path (p8.south)+(0.0,-1.0) \convolution{9}{S:3x3 F:128};
  \path (p9.south)+(0.0,-1.5) \convolution{10}{S:3x3, F:128};
  \path (p10.south)+(0.0,-1.0) \convolution{11}{S:3x3 F:128};
  \path (p11.south)+(0.0,-1.5) \convolution{12}{S:3x3, F:128};
  \path (p12.south)+(0.0,-1.0) \convolution{13}{S:3x3 F:128};
  \path (p13.south)+(0.0,-1.5) \convolution{14}{S:8x8 F:256};
  \path (p14.south)+(0.0,-1.0) \full{15}{256 nodes};
  \path (p15.south)+(0.0,-1.0) \out{16}{5 classes probabilities};
  
             
  % Draw arrows between elements
  \path [line] (p1.south) -- node [above] {} (p2);
  \path [line] (p2.south) -- node [above] {} (p3);
  \path [line] (p3.south) -- node [above] {} (p4);
  \path [line] (p4.south) -- node [above] {} (p5);
  \path [line] (p5.south) -- node [above] {} (p6);
  \path [line] (p6.south) -- node [above] {} (p7);
  \path [line] (p7.south) -- node [above] {} (p8);
  \path [line] (p8.south) -- node [above] {} (p9);
  \path [line] (p9.south) -- node [above] {} (p10);
  \path [line] (p11.south) -- node [above] {} (p12);
  \path [line] (p12.south) -- node [above] {} (p13);
  \path [line] (p13.south) -- node [above] {} (p14);
  \path [line] (p14.south) -- node [above] {} (p15);
  \path [line] (p15.south) -- node [above] {} (p16);    
   
  \background{p3}{p1}{p4}{p3}{FS: 512x512}
  \background{p3}{p4}{p4}{p5}{FS: 256x256}
  \background{p3}{p6}{p4}{p7}{FS: 128x128}
  \background{p3}{p8}{p4}{p9}{FS: 64x64}
  \background{p3}{p10}{p4}{p11}{FS: 32x32}
  \background{p3}{p12}{p4}{p13}{FS: 16x16}
  \background{p3}{p14}{p4}{p15}{Classification}

\end{tikzpicture}
\end{document} 